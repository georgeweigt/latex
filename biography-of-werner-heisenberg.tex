% https://www.heisenberg-gesellschaft.de/biography-of-werner-heisenberg.html

\documentclass{article}
\usepackage{amsmath}
\title{Biography of Werner Heisenberg}
\author{Helmut Rechenberg}
\date{}
\pdfinfoomitdate=1
\pdftrailerid{}
\begin{document}

\maketitle

\tableofcontents

\section{Youth}

Werner Karl Heisenberg was born in Würzburg, Germany, on 5 December 1901. His father, August Heisenberg (1869--1930), stemmed from a family of master craftsmen in Osnabrück, and had studied classical philology in Munich; at the time of Werner's birth he held the dual positions of teacher at the Altes Gymnasium in Würzburg and Privatdozent for Greek philology at the University of Würzburg. His mother was Anna Wecklein, the daughter of Nikolaus Wecklein (1843--1926), a classical philologist and rector of the Maximilians-Gymnasium in Munich. They married in 1899 and their first son, Erwin, was born in Munich in 1900; he later became a chemist and worked in industry (he died in 1965).

The family returned to Munich in April 1910 when August Heisenberg was called (in January) to succeed his teacher Karl Krumbacher in the chair for medieval and modern Greek philology at the university. On 18 September 1911 Werner entered the Maximilians-Gymnasium, still under his grandfather's rectorship and at that time the best humanistic gymnasium in the Bavarian capital. Werner studied diligently and rapidly and was regarded as an outstanding, ambitious, self-confident pupil. He excelled particularly in mathematics; far exceeding the teaching program, he taught himself differential and integral calculus, worked with elliptic functions, and studied abstract number theory. Besides his academic achievements Heisenberg learned to play the piano and by the age of 13 he was playing master compositions. He remained an excellent player throughout his life.

The First World War and its consequences intruded into Heisenberg's boyhood life. His father, a reserve infantry officer, was immediately called to duty and remained away from the family for nearly the entire war. An increasing shortage of food and fuel prevailed throughout Germany and forced the occasional closing of schools. Because of his family's lack of food, Werner signed up for the war assistance service in the summer of 1918 and helped to bring in the harvest with schoolmates on a farm near Miesbach in Upper Bavaria.

The loss of the war and the abdication of the monarchy generated revolutionary unrest throughout Germany. In Bavaria a socialist republic came to power on 7 November 1918; it developed (in Munich) into a Soviet Republic (Räterepublik) on 7 April 1919 that was crushed in early May 1919 by troops dispatched by the Reich Government During the fighting in Munich and the ensuing restoration of moderate socialist rule, Heisenberg, along with many of his schoolmates, voluntarily served in support of one of the republican units, Cavalry Rifle Command No.~11 (Kavallerie-Schützen-Kommando No.~11). He carried messages, guarded the street from the roof of the Catholic seminary across from the university, and read Plato's Timaeus in Greek. Shortly afterwards Heisenberg became associated, as the elected leader of a small group younger boys, with the New Boy Scouts (Bund deutscher Neupfadfinder), a representative of the postwar German youth movement that strove for renewal of personal and social life in Germany. Until the prohibition of independent youth groups in 1933 Heisenberg spent nearly all of his leisure time with his group, hiking and camping within Germany and on trips to neighboring countries (e.g.~to Finland in summer 1923). He loved hiking, skiing, mountain climbing, and enjoyed the beauty of nature all his life.

\section{Student and Postdoctoral Years (1920--1925)}

After a brilliant performance on his final gymnasium examinations (Abitur), for which he was accepted as a scholar of the prestigious Maximilianeum Foundation, Heisenberg entered the University of Munich during the winter semester of 1920/21. At first he planned to study pure mathematics, but a discomforting meeting with the famous professor of that subject, Ferdinand von Lindemann, resulted in his turning to theoretical physics. Arnold Sommerfeld, professor of that field, immediately recognized his talents and admitted him to his Seminar, composed of advanced students and postdoctoral researchers. Among the Seminar members at that time were Gregor Wentzel and Wolfgang Pauli. Otto Laporte and Karl Bechert arrived a little later.

During his three years of university study Heisenberg attended Sommerfeld's lectures covering any field of theoretical physics, including the quantum and relativity theories, as well as the special lectures offered by Karl Herzfeld. He showed less interest in the lectures and laboratory exercises in experimental physics provided by Wilhelm Wien. For one of his two minors he attended mathematics courses offered by Lindemann, Alfred Pringsheim, Artur Rosenthal and Aurel Voss. For his second minor he studied astronomy with Hugo von Seeliger.

In Sommerfeld's Seminar Heisenberg studied the latest literature in physics and carried on intensive discussions of the problems arising from it with his teacher and a group of excellent fellow students. Under Sommerfeld's direction he delved into the problems of atomic theory, a difficult and abstract discipline that tended to attract the best and most dedicated young theorists of the day. As early as his first semester Heisenberg presented a quantum-theoretical analysis of the anomalous Zeeman effect, the results of which he utilized a year later in his first publication (submitted in November 1921). At the same time he examined problems in hydrodynamics, a subject on which he published his second paper (on Kármán vortices, 1922) held his first public lecture in September 1922 (at a topical conference in Innsbruck), and wrote his doctoral dissertation (finished in July 1923).

In June 1922 Sommerfeld brought his star pupil to Göttingen where Niels Bohr presented a series of lectures on quantum theory and atomic structure---the “Bohr Festival” (12--22 June 1922). There Heisenberg met the leading representatives of atomic physics in Germany and Europe, among them Max Born, Paul Ehrenfest, James Franck, Hendrik Anthony Kramers and Alfred Lande. Bohr and Born began to take notice of the young physicist. When Sommerfeld left for a guest professorship at the University of Wisconsin, Madison, during the winter of 1922/23, Born invited Heisenberg to Göttingen as his personal assistant (November 1922--March 1923). In Göttingen Heisenberg learned the rigorous mathematical methods of the Hilbert school and worked with Born on the many-body problem in atomic theory, especially on the energy states of the helium atom.
During the summer semester of 1923 Heisenberg completed his dissertation in Munich. In it he successfully treated the problem of the onset of turbulence, which had resisted all previous efforts made by mathematicians and physicists. Sommerfeld was very pleased with this result and the mathematical methods Heisenberg employed. However, Wilhelm Wien, the examiner in experimental physics, came to quite a different conclusion in his field during the final orals (on 23 July 1923) and wanted to fail the candidate. Heisenberg thus received his doctorate “cum laude” rather than “magna” or “summa cum laude”, the two highest marks. In October 1923 Heisenberg became Born's assistant at the University of Göttingen. He continued to work with Born on atomic and molecular models. By means of a modification of the Bohr-Sommerfeld quantum rules he obtained certain advances in the expIanation of the anomalous Zeeman effect. He got his Habilitation at the University of Göttingen on 28 July 1924. Although only 22 years old, he thus became a Privatdozent and fully qualified as a university lecturer.

Heisenberg made his first visit to Bohr's institute in Copenhagen during the spring of 1924 (15 March to early April). With the help of a Rockefeller fellowship (International Education Board) he resided in Copenhagen during the seven months from middle of September 1924 to early April 1925, and during the fall of 1925 (September--October), and otherwise returned to Born's institute in Göttingen. At Bohr's institute Heisenberg met a number of talented young physicists from a variety of nations, among them Christian Møler (Denmark), Svein Rosseland (Norway), G.H. Dieke (Holland), Ralph Fowler (England), Ralph de Laer Kronig and David Dennison (both from U.S.A.). During his visits he learned to speak Danish and English and worked intensively with Bohr and his closest collaborator Kramers on the most difficult problem of atomic theory. This work led to Heisenberg's papers on resonance fluorescence (submitted in November 1924), the dispersion of light by atoms (with Kramers, December 1924) and the structure of complex spectra and their Zeeman effects (April 1925).

During his years of study and postdoctoral research Heisenberg had received in Munich and Göttingen with Sommerfeld and Born a thorough education in all aspects of theoretical physics, while simultaneously familiarizing himself with the main problems of atomic and quantum theory. With Bohr in Copenhagen he deepened his understanding of the foundations of quantum physics. As his friend Pauli wrote (to Kramers on 27 July 1925), Heisenberg “learned a little of philosophical thinking” and moved “noticeably away from the purely formal”. Heisenberg himself later summarized the influence of his teachers thus: “From Sommerfeld I learned optimism, from the Göttingen people mathematics, and from Bohr physics”.

\section{The Development of Quantum Mechanics (1925--1927)}

The early 1920s witnessed fundamental difficulties in atomic physics. The quantum theory of atomic structure, founded by Bohr and largely developed by Bohr and Sommerfeld, did not describe the properties of complicated atoms and molecules. Moreover, the discovery of the Compton effect at the end of 1922 focussed attention on the problem of the nature of radiation. Its interpretation in the light-quantum hypothesis contradicted classical radiation theory, and the radical attempt by Bohr, Kramers and Slater in early 1924 to resolve the difficulty by assuming only statistical conservation of energy and momentum was refuted by the experiment of Walther Bothe and Hans Geiger in April 1925.

Heisenberg was growing ever more concerned with these and other difficulties in atomic theory. His works on the anomalous Zeeman effect, only successful in part, and his unsuccessful calculation of the helium states with Born had sensitized him by early 1925 to the “crisis” of current theory. Nevertheless, his latest calculations in Copenhagen on dispersion theory and on complex spectra, especially the principle of “sharpened” correspondence applied in these works, seemed to point toward a future satisfactory theory.

With characteristic optimism the Göttingen Privatdozent took on a new and difficult problem at the beginning of May 1925, the calculation of the line intensities in the hydrogen spectrum. Heisenberg began with a Fourier analysis of the classical hydrogen orbits, intending to translate them into a quantum theoretical scheme---just as he had done with Kramers for the dispersion of light by atoms. But the hydrogen problem proved much too difficult, and he replaced it with the simpler one of an anharmonic oscillator. With the help of a new multiplication rule for a quantum-theoretical Fourier series he succeeded in writing down a solution for the equations of motion for this system. On 7 June 1925 he went to the island of Helgoland to recover from a severe attack of hay fever. There he completed the calculation of the anharmonic oscillator, determining all the constants of the motion. He made use, in particular, of a modified quantum condition that was later called by Born, Pascual Jordan and himself a “commutation relation”, and he proved that the new theory yielded stationary states (conservation of energy). Returning to Göttingen on 19 June 1925 Heisenberg composed his fundamental paper “Über die quantentheoretische Umdeutung kinematischer und mechanischer Beziehungen” (“On a Quantum Theoretical Reinterpretation of Kinematic and Mechanical Relations”), which was completed on 9 July 1925. In this paper, the starting point for a new quantum mechanics, Heisenberg announced as the leading philosophical principle of quantum mechanics that only observable quantities are allowed in the theoretical description of atoms. Heisenberg reported his new results during visits shortly thereafter with Paul Ehrenfest in Leiden and with Ralph Fowler in Cambridge.

After Born and Jordan managed in August and September 1925 to develop the mathematical content of Heisenberg's work into a consistent theory with the help of infinite Hermitian matrices (Z. Phys. 34, 858, 1925), Heisenberg participated, starting in September 1925, in the completion and application of the new “matrix mechanics”, culminating in the long “three-man-paper”, by Born, Heisenberg and Jordan, submitted on 16 November 1925. Further developments followed rapidly: Pauli calculated the stationary states of the hydrogen atom in October 1925; Cornelius Lanczos in Frankfurt and Born and Norbert Wiener in the USA extended the method of operator mechanics to describe continuous motions (December 1925); and Paul Adrien Maurice Dirac in Cambridge developed independently of the Göttingen school a different scheme based upon Heisenberg's July paper, the method of q-numbers (November 1925), in which many-electron atoms and the relativistic Compton effect could be handled successfully (spring 1926). In addition Heisenberg and Jordan utilized electron spin and matrix mechanics to solve the old problems of hydrogen fine structure and the anomalous Zeeman effect (April 1926); and finally Heisenberg discovered the phenomenon of quantummechanical resonance (June 1926), which played a decisive role in his subsequent calculation of the term system of the helium atom (July 1926).

In May 1926 Niels Bohr offered Heisenberg a position at his institute in Copenhagen as Lector and successor to his assistant Kramers. There Heisenberg delivered lectures at the university (in Danish) on contemporary physical theories, directed beginning students, helped guest researchers with their problems, and discussed with Bohr the most important results of quantum mechanics. In the summer and fall of 1926 the main topic of discussion was wave mechanics, the quantum atomic theory that Erwin Schrödinger began introducing in January 1926. The complete mathematical equivalence between Göttingen's matrix mechanics or Dirac's q-number scheme and Schrödinger's wave mechanics was proved by Jordan and Dirac in December 1926, after preparatory work by Schrödinger (March 1926), Pauli (April 1926), and Carl Eckart (June 1926). However, Schrödinger's physical interpretation of the square of the wave amplitude as the continuously distributed charge density of the electron was rejected by Born, Bohr and Heisenberg and replaced on Born's proposal by the interpretation that it is the probability for finding the electron at each location (June 1926). In close contact with Pauli, and in intense discussion with Bohr, Heisenberg analyzed what he termed the “perceptual content of the quantum-theoretical kinematics and mechanics”. As a result of the analysis he presented in March 1927 the so-called “indeterminacy” or “uncertainty relations”, which limit the simultaneous measurement of canonically conjugate variables, such as the position and momentum of a particle. Bohr, on the other hand, pondered the simultaneous use of the physical pictures of particles and waves, which resulted in his general principle of “complementarity” announced in fall of 1927.

Born's statistical interpretation of Schrödinger's wave function, Heisenberg's uncertainty relations, and Bohr's complementary principle formed the basis of the physical interpretation of the new quantum mechanics, as explicated by Bohr in his lectures at the Volta Conference in Como (September 1927) and at the Solvay Congress in Brussels (24--29 October 1927). This “Copenhagen Interpretation” of quantum mechanics, as it was later called, found acceptance by most physicists, but not by all: Albert Einstein in particular raised serious objections to it at the 1927 and 1930 Solvay conferences and later, for example, in his paper with Boris Podolsky and Nathan Rosen (Phys.~Rev.~47, 777, 1935).

\section{Professor in Leipzig (1927--1942)}

Through his fundamental contributions to quantum mechanics Heisenberg soon became known far beyond the circle of the Munich, Göttingen and Copenhagen physicists as a leading representative of the new, successful atomic theory. At the Meeting of the German Scientists and Physicians in Düsseldorf(19--25 September 1926) he delivered the main address on quantum mechanics. In October 1927 Heisenberg, then only 25 years old, was called from Copenhagen to the chair for theoretical physics at the University of Leipzig, where Peter Debye had just accepted the chair for experimental physics. In spring 1929 Friedrich Hund joined Heisenberg's theoretical physics institute, and in 1931 the mathematician Bartel Leendert van der Waerden, who developed group theoretical methods in quantum mechanics, came to Leipzig as professor of mathematics.

Together Heisenberg and his colleagues created in Leipzig a leading center for research on atomic and quantum physics that attracted numerous talented students and collaborators. Felix Bloch, Rudolf Peierls and Edward Teller were among Heisenberg's first students. Hans Euler, Leon Rosenfeld, Victor Weisskopf, Carl Friedrich von Weizsäcker and Gian Carlo Wick also studied under Heisenberg. As part of their research program Debye and Heisenberg set up an annual symposium (until 1933) on current research in physics, the famous Leipzig University Week (Universitätswoche). The first of these took place on 18--23 June 1928 with, among others, Paul Dirac speaking on his relativistic electron equation and Fritz London on the theory of the chemical bond.

Heisenberg's research in Leipzig concentrated first on applications and extensions of quantum mechanics. In May 1928 he provided an explanation of ferromagnetism from first principles in which he showed that the quantum mechanical exchange integral that had played a decisive role in his solution of the helium problem could account for the strong molecular magnetic field in the interior of ferromagnetic materials. A little later Felix Bloch, supplemented Heisenberg's theory of ferromagnetism with the invention of spin waves. During the period 1928--1931 Heisenberg and his pupils presented a series of basic contributions to the theories of molecules and metals. At the same time Heisenberg collaborated with Wolfgang Pauli, who held the chair in theoretical physics at the E.T.H. (Swiss Polytechnic) in  Zürich since spring 1928, on the formulation of relativistic quantum field theory. In March and September 1929 they completed two basic papers that allowed a systematic treatment of interacting quantum fields, but that also indicated the existence of severe difficulties and divergences in the theory, such as infinite self energies of electrons. The discovery of the neutron in early 1932 occasioned Heisenberg's development of the neutron-proton model of the nucleus (independently of Dimitri Iwanenko), in which he introduced the concept of nuclear exchange force and the formalism of isotopic spin(June and July 1932). In recognition of his work on quantum mechanics and its applications Heisenberg received the Nobel Prize for physics in 1933 (for the year 1932).

Besides his intensive research during the early years in Leipzig, Heisenberg undertook a number of trips abroad. In spring of 1929 he delivered a series of lectures at the University of Chicago on “The Physical Principles of the Quantum Theory” which were published as his first book. He made the return trip with Dirac, who had also come to the United States, through the American West (August 1929) to Japan and India (September--November 1929). In 1932 he lectured at the Ann Arbor summer school. Heisenberg participated as well in numerous conferences, such as the Solvay congresses of 1927 (24--29 October), 1930 (20--25 October) and 1933 (22--29 October) in Brussels, the centenary meeting of the British Association in London(29 September 1931) and the nuclear physics conference in Rome (October 1931).

Throughout the 1930s Heisenberg very frequently visited Niels Bohr in Copenhagen and occasionally Wolfgang Pauli in Zurich; his close relations with both men and their colleagues were for him irreplaceable in the difficult period.

Quantum mechanics led to entirely new conceptions in the description of the microscopic world and to a revision of some basic physical and philosophical notions, such as the principle of causality. After the completion of the theory by Born, Bohr and Heisenberg, the latter two, as did others, devoted considerable attention to obtaining many of the intellectual consequences of the new theory. These consequences often went far beyond the realm of physics, extending into chemistry, biology, and even into social and ethical phenomena. To many laymen, however, the new ideas seemed as strange and disturbing as the revisions brought about earlier by the relativity theory---if not more so. In order to explain the new conceptions to nonspecialists and to relieve the public of any bewilderment, which---as with relativity theory---grew dangerously infected with ideology, Heisenberg wrote for and spoke often before wide audiences of laymen interested in science during the 1930s. His widely read first collection of essays, Wandlungen in den Grundlagen der Naturwissenschaft (1935), contained two such addresses.

Despite the political situation in Germany, which also affected his collaborators, Heisenberg's physics continued to flourish. The main focus of his research concentrated during the 1930s on the analysis of problems in relativistic quantum field theory. A major component of this work involved the analysis of elementary processes in high energy cosmic-ray interactions, where the new elementary particles positron and meson were discovered in 1932 and 1937, respectively. For the solution of problems in field theory and the explanation of cosmic-ray phenomena, fundamentally new ideas appeared to Heisenberg unavoidable. In cosmic-ray physics he argued in favor of the existence of particular multiple processes, which he called “explosion showers”, that could be comprehended only by means on a nonlinear quantum field theory (June 1936, May 1939). He further claimed that the divergence difficulties in quantum field theory had to be resolved by assuming a fundamental length that limited the validity of field theory at very small distances (June 1938).

In the years following 1933 Heisenberg became the main spokesman in Germany for modern theoretical physics. Although the number of his students decreased, Heisenberg presented systematic lectures on all fields of theoretical physics, including the politically disfavored relativity---both the special and the general theories. He defended relativity theory even in an article published in the Nazi Party newspaper Völkischer Beobachter on 26 February 1936. Together with two colleagues in experimental physics, Hans Geiger in Berlin and Max Wien in Jena, Heisenberg authored a memorandum to the Reich Minister for Education and Science (Reichsminister für Erziehung und Wissenschaft) opposing the official discrediting of theoretical physics and emphasizing its great importance for the training of needed physicists. His opponents, who gathered around Philipp Lenard and Johannes Stark, sought to dismiss relativity and quantum theory as “Jewish” and “non-German”.

After Arnold Sommerfeld reached retirement age in 1934, attacks by representatives of the so-called “German physics” greatly increased when Heisenberg was named by the faculty of the University of Munich as the first candidate to succeed him. On 25 July 1937 an article entitled “Weiße Juden in der Wissenschaft” (“White Jews in Science”) and signed by Stark appeared in Das schwarze Corps, the journal of the SS. It contained wild accusations against modern theoretical physics and theoreticians, such as Heisenberg, and closed with the demand that Heisenberg and his colleagues be made to “disappear” like the Jews. Heisenberg responded with a letter to the head of the SS, Heinrich Himmler, urgently requesting a termination of the campaign against him. After a long series of investigations and interrogations by the SS, Heisenberg attained his goal in a letter from Himmler on 21 July 1938. The campaign ceased, but the Munich chair went in December 1939 to the aerodynamicist Wilhelm Müller. Heisenberg managed, however, to have an article on the significance of modern theoretical physics published in the organ of the national students union (Reichsstudentenbund). The article, entitled “Die Bedeutung der 'modernen' theoretischen Physik” (“The meaning of 'modern' theoretical physics”), finally appeared in 1942 after considerable delay, and was eagerly read by many young people.

In spite of the official discrediting of theoretical physics in the Third Reich, Heisenberg and Hund managed to keep an unusually stimulating atmosphere in their institute, made even more lively by many foreign guests. Besides the German members of the institute (E.~Bagge, S.~Flügge, B.~Kockel and H.~Voltz), there were a number of visitors from America (R.S.~Mulliken, J.C.~Slater, J.H.~Van Vleck, C.~Zener), Italy (U.~Fano, G.~ Gentile), Japan (K.Ariyama, Y. Fujioka, S. Kikuchi, S. Tomonaga, K.Umeda, S. Watanabe), Scandinavia (B.O. Gronblom, H.~Wergeland), Hungary (L.~Tisza) and Yugoslavia (I.~Supek), to name only a few. Heisenberg also continued his relations with foreign physicists through lectures and travels to scientific conferences. In 1937 he attended the Galvani Conference in Bologna (18--21 October), in April 1934 and March 1938 he traveled to England (to Cambridge and Manchester, respectively), and in summer of 1939 he went to America where he presented his latest results on the theory of cosmic-ray showers at the Symposium on Cosmic Rays at the University of Chicago (27--30 June). Although he received several tempting offers of positions in America, he decided to return to Germany.

In April 1937 Heisenberg married Elisabeth Schumacher, daughter of Hermann Schumacher, a professor of economics at the University of Berlin. In early 1938 the twins Maria and Wolfgang were born; the family was later increased by Jochen (1939), Martin(1940), Barbara (1942), Christine (1944) and Verena (1950). For a secure place of refuge during the imiainent war the Heisenbergs bought in 1939 a summer house in Urfeld on Lake Walchen, south of Munich, that had previously belonged to the painter Lovis Corinth. When Heisenberg transferred to Berlin in 1942 to assume directorship of the Kaiser Wilhelm Institut für Physik, his family did not accompany him but moved to Urfeld rather than to Berlin, for safety against bombing raids.

\section{The War Years (1939--1945)}

The outbreak of war on 1 September 1939 directly affected Heisenberg's scientific career. He was called to military service, and to his surprise was ordered to report to the Army Weapons Bureau (Heereswaffenamt) in Berlin. There he and other leading German atomic scientists, the so-called Uranium Club (Uranverein), were asked to investigate whether the fission, of uranium, discovered in December 1938 by Otto Hahn and Fritz Strassmann in Berlin, could be used for gaining energy on a large scale. Within two months Heisenberg completed a comprehensive report in which he developed the theory of a chain reaction with uranium fission by neutrons ( 6 December 1939); he followed it soon by a second report refining the theory and taking into account recently improved data (29 February 1940).

The Army Weapons Bureau designated the Kaiser-Wilhelm-Institut (KWI) für Physik in Berlin-Dahlem as the center of its uranium research. Peter Debye, director of the institute and a Dutch citizen, was prevented as a foreigner from working on the secret research. He placed himself on leave and accepted a guest professorship at Cornell University in January 1940. Kurt Diebner assumed Debye's office as head of the Berlin uranium project. At the suggestion of C.F. von Weizsäcker and especially Karl Wirtz, two of Debye's former collaborators now working on the project, Heisenberg acted as scientific advisor. He thus traveled weekly to Berlin from Leipzig and wrote several reviews on the progress of research at the Berlin institute (December 1940--February 1942). Other, non-secret research begun under Debye (for example on X-ray analysis and on nuclear and low temperature physics) continued at the KWI für Physik under the formal supervision of Max von Laue, vice-director. During 1941 and 1942 Heisenberg also directed a seminar at the institute in which he and other members and guests of the institute discussed the recent progress and problems of cosmic-ray physics. These talks were published in 1943 in the collection Kosmische Strahlung (Cosmic Radiation) dedicated to Sommerfeld on his 75th birthday.

In spring of 1940 Heisenberg began experiments with Robert and Klara Döpel at the University of Leipzig in which they studied possible arrangements of uranium and neutron moderator substances, since for uranium fission slow neutrons turned out to be most effective. These experiments substantiated the usefulness of heavy water as moderator (28 October 1941) and indicated in early 1942 that a spherical arrangement of natural uranium and heavy water led to a small multiplication of the irradiating neutrons (confirmed in June 1942). Thus the prerequisites for the construction of a functioning reactor were fulfilled. On 4 June 1942 a meeting was held between Albert Speer, Minister for Armament and War Production (Reichsminister für Rüstung und Kriegsproduktion), and other officials and military leaders and the scientists, including O. Hahn, W. Heisenberg, K.~Diebner, P.~Harteck and K.~Wirtz. There it was decided that the uranium project should continue with the goal of constructing a nuclear reactor, but not a nuclear bomb. Isotope enrichment of U-235---which would be necessary for a bomb---was not considered. As a result, the Army Weapons Bureau returned the KWI für Physik to the control of the civilian Kaiser Wilhelm Society (Kaiser-Wilhelm-Gesellschaft), but the reactor project remained secret. Heisenberg, the chief scientific advisor to the institute since 1940 and theoretical director of the successful Leipzig experiments, was named head of the leading reactor research group in Germany and appointed on July 1942 director at the KWI für Physik (to substitute for the absent director Debye). He was assisted by many of Debye's assistants, including Horst Korsching and Karl Wirtz, as well as by Erich Bagge and Fritz Bopp. Kurt Diebner, previous head of the Berlin project, went with several collaborators to the army laboratory at Gottow where he set up his own uranium experiments. C.F.~von Weizsäcker accepted a professorship for theoretical physics at the University of Strasbourg.

The work on the uranium project did not exhaust Heisenberg's activities during the war years. He occupied himself particularly with problems in cosmic-ray physics and developed in several papers a new approach to the theory of elementary particles based on the concept of the scattering matrix. Three papers, received in September and October 1942 and May 1944, respectively, were published in the Zeitschrift für Physik. He regularly gave lecture courses at the universities of Leipzig and Berlin (after summer 1942) on topics of theoretical physics. And he traveled, as far as the war restrictions allowed, to foreign countries, e.g. to Budapest (April 1941), to Zurich and Bern(November 1942) and to Leiden and Utrecht(October 1943), to meet with colleagues and friends and to deliver talks on his (non-secret) scientific research or on more general questions of physics and science. A particular visit was paid in October 1941 to Niels Bohr in Copenhagen. Without violating the secrecy of his work on the uranium project, Heisenberg tried to convey the message that in Germany the construction of the atomic bomb was not being considered. The failure of this visit cast a shadow on the future relations between Heisenberg and his teacher and friend.

While Fermi at Chicago in his reactor achieved a critical chain reaction already in 1942, progress in the German project was very slow. Although the basic scientific and technological problems had been overcome and only the task remained of assembling enough uranium and heavy water to create an energy producing reactor, the project was increasingly hindered by the war conditions. Germany could barely provide enough uranium and not nearly enough heavy water, deficiencies made more acute by the fact that at least three groups performed competing experiments: besides the largest group in Berlin, a group in Hamburg under Paul Harteck and the group in Gottow under Diebner. The increasing air raids on Berlin forced the transfer of all apparatus to a bunker in the Dahlem institute, and finally, in late 1944, the entire project was shipped out of Berlin and housed in a rock bunker under the castle chapel in the southern German town of Haigerloch. The KWI für Physik had already been transferred earlier (since 1943) to the nearby town of Hechingen. In the first months of 1945 the model reactor B8 nearly went critical (the point at which the reaction is self-sustaining). But the lack of a few hundred kilograms of uranium (out of 1.5 tons) and of several hundred liters of heavy water prevented success before war's end.

At the end of April 1945 members of an American science intelligence unit within the Manhattan Project, the Alsos Mission, led by Colonel Boris T. Pash and the Dutch physicist Samuel A. Goudsmit, reached Tailfingen (where Otto Hahn's KWI für Chemie had moved), Hechingen and Haigerloch just ahead of the advancing French troops. The German atomic scientists gathered there---Bagge, Hahn, Korsching, von Laue, von Weizsäcker and Wirtz---were taken prisoner. Heisenberg, however, had gone a few days earlier by bicycle to Urfeld to join his family. He was taken prisoner there on 3 May 1945 by a small unit of the Alsos Mission led by led by Colonel Boris T.~Pash. Heisenberg and Walther Gerlach, Reich Commissioner (Reichsbeauftragter) for the reactor project, along with Diebner and Harteck, soon joined the Tailfingen-Hechingen group in France. All ten scientists were transferred to Castle Facquerel in Belgium and finally to England (on 23 July 1945), while the Americans dismantled the Haigerloch reactor and transported all of the apparatus, uranium, heavy water, and numerous documents to the United States.

\section{The Period of Reconstruction and Renewal (1946--1958)}

In April 1945 Soviet troops reached Berlin, the Reich collapsed and the Russians began sending to the Soviet Union the remaining apparatus in the Dahlem institute, the institute library, and the several German scientists they found there (among them Ludwig Bewilogua who had built the low temperature laboratory and had run the high voltage generator during the war). The Berlin KWI für Physik ceased to exist, while the Hechingen division barely survived with a handful of scientists---including Fritz Bopp, the crystallographer Georg Menzer and the spectroscopist Hermann Schüler---and a handful of experiments unrelated to nuclear physics.

Heisenberg was held along with the other prominent members of the Uranium Club at the country estate Farm Hall near Cambridge, England, well treated and equally well watched. During their stay at Farm Hall the atomic bomb was dropped on the Japanese city of Hiroshima on 6 August 1945. From news reports the German scientists learned that the Allies had, in their efforts to gain energy from atomic fission, obviously progressed much further than the Germans and had achieved military uses, an application that the Germans had believed to lie only far in the future. Their many discussions about the consequences of military applications of nuclear energy contributed to their later stance against nuclear weapons for the German Army.

At Farm Hall Heisenberg also set the stage for his future research. After years of difficulty, political pressure and reduced exchange of ideas he took the opportunity to devote himself in leisure to a wide range of physical problems. Included were the still unresolved problems of explaining superconductivity, which he discussed with von Laue, and the application of statistical methods to the description of turbulence, discussed with von Weizsäcker. He also decided to remain in Germany, despite lucrative offers to come to the United States and to work at the KWI für Physik if that would become possible.

On 3 January 1946, the German scientists were released from Farm Hall and returned to Western Germany. Heisenberg went with Hahn, von Laue, von Weizsäcker and Wirtz to Göttingen in the British Occupation Zone, where Max Planck and Ernst Telschow were seeking to continue the Kaiser Wilhelm Society and its institutes. Heisenberg devoted himself upon his return to two large tasks: the reconstruction of the KWI für Physik as a center for experimental and theoretical research in physics and the renewal of research in Germany. In particular he pondered a suitable vehicle for mediating between scientific research and the future German government in order to insure technical prosperity and the protection of science as the source of technical progress.

During the immediate post-war years Germans had to rebuild practically from the ground up. With the collapse of economic and political organization and the division of Germany into four occupation zones, the reconstruction of research in Germany was extremely difficult. The conditions were perhaps most favorable in the British Zone, where the authorities, especially the liaison officer Colonel Bertie Blount, appeared quite open to the needs of the German scientists. As early as 1 January 1946 scientists and officials formed the German Scientific Council (Deutscher Wissenschaftlicher Rat) under the chairmanship of the Nobel Prize chemist Adolf Windaus. It mediated between the British military government and German scientific institutes and enabled, among other activities, the rebuilding of the Physico-Technical-Institute (Physikalisch-Technische Reichsanstalt) in Braunschweig and the refounding of the German Physical Society (Deutsche Physikalische Gesellschaft) within the British Zone. The old British Hanoverian university city of Göttingen, largely untouched by the war, became once again a scientific center. During a meeting with Göttingen scientists in October 1946 the British authorities permitted the installation in Göttingen of several Kaiser Wilhelm institutes on the grounds of the former Aerodynamical Research Institute (Aerodynamische Versuchsanstalt) which had been completely dismantled as war-related. The institutes for physics, physical chemistry and medical research, previously all in Berlin, were thus revived. Finally the Kaiser Wilhelm Society was renamed and reinstituted in the three western zones as the Max Planck Society with seat in Göttingen; after 16 February 1948 all its institutes in western Germany were renamed Max Planck institutes (MPI).

In July 1946 Heisenberg began reequipping the institute for physics, whose direction he assumed after the official director Debye declined an offer to return. Two departments were set up initially: one for theoretical physics under von Weizsäcker, the other for experimental physics under Wirtz. Research was limited by the directives of the Allied Control Commission in that Germans were forbidden to work in a variety of fields considered war related, especially areas of nuclear physics involving slow neutrons and nuclear fission. Despite the limitations Heisenberg and his colleagues, many of whom had worked with him earlier in Berlin, did not lack research topics. Foremost remained the study of cosmic-ray physics, begun in Leipzig, continued in Berlin, and now picked up once again in Göttingen. Most of the institute resources were directed at first into theoretical research, since financing did not suffice for experimental studies. Only gradually did the institute catch up with the rapid developments occurring elsewhere in cosmic-ray and elementary particle physics at that time. The second edition of Kosmische Strahlung, published in 1953, indicated a near approximation to international standards in this subject. During the early post-war years Heisenberg also presented a theory of superconductivity (1946--1948), a statistical theory of turbulence (1946--1948), and examined the theoretical properties of elementary particles (after 1946).

Besides the direction of his institute and his own research Heisenberg devoted himself with great energy to the revival of scientific research in Western Germany and especially to his conception of governmental science policy. Learning from his experiences in the Third Reich, his observations of the close cooperation between scientists and government in Great Britain, and believing that a modern industrial state required a central science policy, Heisenberg sought a direct role for the government of the new Federal Republic of Germany in forming a national policy of support for science and technology and a role for science advisors to the chancellor. Such conceptions found expression in the German Research Council (Deutschen Forschungsrat) founded on 9 March 1949 by the Max Planck Society and the surviving academies of science in West Germany. It was composed of 15 leading scientists with Heisenberg as president. The new council, supported by Chancellor Konrad Adenauer, represented Ger man science in international affairs and directly in the chancellor's office. Among its successes were the acquisition of Marshall Plan money for the support of German research, the admission of the Federal Republic to the International Union of Scientific Councils and to UNESCO, the provision of federal responsibility for science in the West German constitution, and the assurance of mutual support between German industry and science. Yet the institution of the German Research Council contradicted the long tradition in Germany that science fell under the authority of the cultural ministers of each state (Land). Hence the council was challenged increasingly by the Emergency Association of German Science (Notgemeinschaft der Deutschen Wissenschaft), revived in January 1949 by the ministers of culture and the university rectors. To avoid rivalry and the damaging overlap of interests, Heisenberg, with much regret, gave up his idea of a federal science body. The two bodies, e.i. the Research Council and the Emergency Association, were joined into the present-day German Research Association (Deutsche Forschungsgemeinschaft) in August 1951. The task of representing German science was assumed by the senate of the new body composed mainly of members of the former Research Council. Heisenberg was elected to the presidential committee of the new Research Association; he also directed after February 1952 its Commission for Atomic Physics (Kommission für Atomphysik) which coordinated nuclear research (with the exception of research in nuclear fission) nationwide.

Heisenberg's constant involvement in German science policy found its counterpart in his involvement in the organization of research in his own institute. On.1 July 1947 a new astro-physical department was added under the direction of Ludwig Biermann, who had earlier worked at the Babelsberg observatory and the University of Hamburg. Biermann and his collaborators concentrated in Göttingen on cosmic plasma physics and on quantum theoretical problems in astrophysics. They soon attained significant results, in particular the prediction of the so-called “solar wind” from an analysis of the behavior of comet tails (Biermann and Rhea Lüst) and a new formation of the basic equations of plasma physics (Arnulf Schlüter). Their work required, as did other topics treated at Heisenberg's institute, the increased application of electronic data handling. Since such tools were not available at that time in Germany, the Max Planck Society began its own development program. Heinrich Billing, since 1 June 1950 head of the computer group at the Institute for Instrument Studies (Institut für Instrumentenkunde) in Göttingen and after 1957 a member of the MPI für Physik developed and constructed the programmed calculating machines Gl and G2; the former was used after October 1952 at the MPI in Göttingen. Billing's breakthroughs in computer technology were unfortunately not realized by German industry.

The work produced by the Göttingen MPI during the years of reconstruction won a growing international reputation, and the institute's director Heisenberg attained increasing recognition as the leading representative of German science in the international arena. The reestablishment of international relations, also of great importance to Heisenberg, began officially, with his invitation to lecture at the British universities in Cambridge, Edinburgh and Bristol in December 1947. During the following years he repeatedly visited Niels Bohr in Copenhagen and in summer 1950 he participated in the International Congress of Mathematicians in Cambridge, Massachusetts (30 August--6 September). In 1954, on a travel to the US, he visited Einstein in Princeton and discussed quantum mechanics with him. He also served as the West German delegate to the conference on Atoms for Peace in Geneva. When in 1952 the European Council for Nuclear Research came into being, Heisenberg headed the German delegation and participated in the decision to locate the large European research center for high energy physics, CERN, in Geneva, Switzerland. The Scientific Policy Committee of CERN, responsible for planning the research program, elected Heisenberg as its chairman.

Immediately after his return to post-war Germany Heisenberg emphasized in a programmatic speech before Göttingen students the role of “science as a tool of mutual understanding between peoples” (13 July 1946). He plunged into the task of getting German scientists reaccepted as members of the international family of scientists and of renewing personal relations after the isolation and alienation in the previous period of the Third Reich and World War II. In this endeavor he assigned a particularly important role to the reestablished Alexander von Humboldt Foundation, whose first president he was appointed on 10 December 1953 by Chancellor Adenauer. The purpose of the foundation is to enable young scholars and scientists from around the world to collaborate with German colleagues while guests at German research institutes. Numerous Humboldt fellows were invited to Heisenberg's institute in Göttingen and later in Munich. He held the president's office until a severe illness forced him to resign in October 1975.

On 5 May 1955 the Paris Accords came into effect, whereby the Western A1lies granted the Federal Republic full sovereignty and full membership in the NATO alliance. All restrictions on West German research were thus removed and, after ten years of restraint, the development of German nuclear energy resumed in full. In October 1955 Adenauer created a Federal Ministry for Atomic Questions (Bundesministerium für Atomfragen), forerunner of the wider ranged, present-day Ministry for Research and Technology (Bundesministerium für Forschung und Technologie). Heisenberg served as a leading member of the German Atomic Commission (Deutsche Atomkommission) set up by the new ministry and composed of scientists, industrialists, and politicians with the purpose of advising the ministry on nuclear energy policy. Heisenberg also directed the committee on nuclear physics within the Atomic Commission, served as a member of the Bavarian Atomic Commission, and acted as the main impetus toward the construction of Germany's first nuclear reactor, a research model set up at Garching near Munich in October 1957.

While Heisenberg supported the development of nuclear energy for peaceful uses, he and other scientists equally energetically opposed the plans of Chancellor Adenauer to equip the West German army with tactical nuclear weapons. Adenauer met with stiff resistance from Hahn, Heisenberg, von Weizsäcker and other atomic scientists, eighteen of whom issued a public declaration (basically formulated by von Weizsäcker and Heisenberg) from Göttingen on 12 April 1957, opposing research on or possession of nuclear weapons by West Germany. The West German army has since remained non-nuclear.

Heisenberg's main scientific interest in the early fifties increasingly focused on the search for a consistent quantum field theory of elementary particles. After an unsuccessful attempt at obtaining a nonlocal theory, he became concerned after 1952 with the investigation of nonlinear field equations, in which the mathematical space of states was extended beyond that used since the early days of quantum mechanics. In this theory, which Heisenberg and his collaborators developed in a series of papers between November 1953 and December 1956, the conditions of relativistic invariance could be immediately introduced and finite results could always be obtained without the use of supplementary subtraction or normalization procedures. After Heisenberg had demonstrated the consistency of his ideas and methods in case of a model field theory, the so-called Lee model (October 1957), he entered into a close collaboration with Pauli that yielded in early 1958 the proposal of a nonlinear spinor equation designed to describe the properties and the behavior of all known elementary particles (dubbed the “world formula” by eager journalists). While Heisenberg expounded the equation in April 1958 and on several later occasions, Pauli withdrew his support.

Heisenberg's efforts to obtain a consistent quantum field theory for all elementary particles harmonized with the philosophical views that he presented at that time in many public lectures and lecture series (e.g., in the Gifford Lectures delivered in the winter term 1955/56 at St. Andrew's University in Scotland).

In September 1958 Heisenberg's MPI für Physik moved from Göttingen into a large new building in Munich near the English Garden, which solved the problem of overcrowding at the old institute. Differing from original plans, Karl Wirtz and his reactor group did not move to Munich, but left the MPI in March 1957 and settled at the nuclear research center of theKernreaktor Bau- und Betriebsgesellchaft mbH. close to Karlsruhe. C.F.~von Weizsäcker did not, go to Munich either, as he accepted in June 1957 a call to a philosophy chair at the University of Hamburg. Yet he remained a regular guest in the Munich MPI during his semester vacations.

\section{The Munich Years (1958--1976)}

When Heisenberg moved to Munich in 1958, he characteristically broadened the program of his institute, which ultimately led to the establishment of a series of new institutes. First, the former astrophysics department under Ludwig Biermann was raised to an Institut für Astrophysik which, together with the Institut für Physik, formed Heisenberg's Max-Planck-Institut für Physik und Astrophysik. The program of the Göttingen institute was divided between the two institutes. The groups for experimental and theoretical elementary particle physics (and related fields) constituted, together with a group on experimental plasma physics (under Gerhard von Gierke), the physics institute. The astrophysics institute housed a department for electronic computers (Billing), one for theoretical astrophysics (Reimar Lüst) and one for theoretical plasma physics (Schlüter). With the addition of experimental plasma physics the institute compensated for the loss of Wirtz' reactor group and kept an open option on contributing to fundamental research in nuclear fusion. Yet it soon overflowed the available laboratory space in Munich and in June 1960 began moving to Garching, to form the present MPI für Plasmaphysik (move completed in1968). Heisenberg remained, however, in the scientific administration of the new institute. The MPI für extraterrestrische Physik under Reimar Lust also emerged (in 1964) from Heisenberg's institute and settled on the grounds of the Garching unit, but it remained an official sub-division of the mother institute.

While Heisenberg played an active role in these extensions of his institute, he remained convinced that the large experimental and theoretical tasks in elementary particle physics could only be solved by international efforts. He thus continued to aid and advise the European laboratory CERN near Geneva, which began operations in January 1959. Many members of his institute performed experimental work at CERN. During the later expansion of CERN Heisenberg actively supported the construction of the storage rings, which he dedicated on 16 October 1971. He also provided essential support during the early planning stages for the electron synchrotron (DESY) near Hamburg, whose first high-energy accelerator went into operation with a 6 GeV electron beam in February 1962.

Among Heisenberg's theoretical researches in Munich the development and evaluation of the nonlinear spinor theory occupied center place. In March 1959 he completed with his collaborators Hans-Peter Dürr, Heinrich Mitter, Siegfried Schlieder and Kazuo Yamazaki a long paper enunciating the principles of the theory together with several consequences; they obtained, in particular, its resonance states, one of which, the $\eta$-meson, was found more than a year later. Further investigations yielded an organizing of nucleons and hyperons with the help of the spurion concept (December 1964) and a calculation of the electromagnetic fine structure coupling constant (January 1965). Heisenberg wrote an introductory textbook on nonlinear spinor theory, which was published in 1966, and he retained his belief in it until the end of his life. He argued that all experimental researches on the structure of matter during the past decades had yielded either the already known elementary particles, or new ones having the same qualities. Moreover, the assumption of successive layers of further, ever more elementary objects, such as quarks, would not allow the achievement of the main goal of a fundamental theory, i.e., to explain the dynamical behavior of matter, it would rather relegate the problem to the next deeper level. It appeared to Heisenberg that his nonlinear spinor theory agreed in spirit with Plato s representation of the structure of matter by simple geometrical forms. He proposed to replace Plato's form by a highly symmetrical field equation. All properties of matter should follow from this field equation and the imposed conditions, although the detailed computations involved nonlinear and often very complicated approximation procedures.

In Heisenberg's later work the foundations of physics, which guarantee the unity of the description of nature, tended to merge with the conceptions of Plato's world view. He tirelessly presented his ideas about the intimate connection between physics and philosophy to wider audiences. He saw this connection verified in the historical development of quantum physics, describing parts of his own role in that development in his recollections Der Teil und das Ganze (1969, in English: Physics and Beyond, 1971). But the connection was also reflected in Heisenberg's standpoint on questions of art, and even on questions of religion and society.

On 31 December 1970 Heisenberg resigned the directorship of the MPI für Physik und Astrophysik the institute that he had directed for almost 30 years. Yet he still came regularly to his office and continued to work on scientific papers and on more general articles, lectures and the second edition of his book on the nonlinear spinor theory. He participated at selected conferences, such as the symposium in honor of Dirac's 70th birthday in Miramare near Trieste (18--25 September 1972) and the colloquium celebrating the 200th anniversary of the Brussels Academy (16--17 May 1973). He traveled to the United States for the last time in April 1973. In the middle of 1973 he fell seriously ill. He slowly improved and a year later he appeared to have fully recovered. But in July 1975 he severed a severe relapse. He died at his home in Munich on 1 February 1976.

Heisenberg received numerous national and international awards in recognition of his work and his influence on science and society. He received, besides the Nobel Prize for physics, the Barnard Medal of Columbia University (New York, 1929), the Matteucci medal of the Accademia Nationale dei Lincei (Rome, 1929) the Planck medal of the German Physical Society(1933), the Copernicus prize of the University of Königsberg (1943), the Hugo Grotius medal (1956), the order Pour le merite für Wissenschaften und Künste (1957), theKulturpreis of the City of Munich (1958), the Niels Bohr medal (Copenhagen, 1970) and the Guardini prize (Munich, 1973). In addition he received high national and international honors. He was a member of more than 30 scientific societies, including the Saxonian Academy (Leipzig, 1930), the Kaiserlich Leopoldinisch-Carolinische Akademie der Naturforscher (Halle, 1933), Norwegian Academy of Sciences (Oslo, 1936), Göttingen Academy of Sciences (1938), the Royal Swedish Academy (Uppsala, 1938), the Société Philomatique (Paris, 1938), the Royal Dutch Academy of Sciences (Amsterdam, 1939), the Prussian Academy of Science (Berlin, 1943), the Accademia Nazionale dei Lincei (Rome, 1947), the Bavarian Academy of Sciences (Munich, 1949), the Royal Danish Academy of Sciences (Copenhagen, 1951), the Pontificial Academy of Sciences (1955), the Royal Society of London (1955), the American Academy of Arts and Sciences (Boston, 1958) and the US National Academy of Sciences (Washington, 1961). At the very beginning of Heisenberg's studies Sommerfeld reminded him of the words of Schiller: ``Wenn Könige bauen, haben Kärrner zu tun.'' (“When kings build, wagoners have to work”) By this he meant that his pupil had first to work along side the wagoners. Little did he suspect how quickly Heisenberg would begin to build and to contribute, as did few others in our century, to the construction of modem theoretical physics.

\end{document}
