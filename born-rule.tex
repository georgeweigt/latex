% https://www.quantamagazine.org/the-born-rule-has-been-derived-from-simple-physical-principles-20190213
\documentclass[12pt]{article}
\usepackage{amsmath}
\pdfinfoomitdate=1
\pdftrailerid{}
\begin{document}

Everyone knows that quantum mechanics is an odd theory, but they don't necessarily know why. The usual story is that it's the
quantum world itself that's odd, with its superpositions, uncertainty and entanglement (the mysterious interdependence of
observed particle states). All the theory does is reflect that innate peculiarity, right?

Not really. Quantum mechanics became a strange kind of theory not with Werner Heisenberg's famous uncertainty principle in
1927, nor when Albert Einstein and two colleagues identified (and Erwin Schrodinger named) entanglement in 1935. It happened
in 1926, thanks to a proposal from the German physicist Max Born. Born suggested that the right way to interpret the wavy
nature of quantum particles was as waves of probability. The wave equation presented by Schrodinger the previous year, Born
said, was basically a piece of mathematical machinery for calculating the chances of observing a particular outcome in an
experiment.

In other words, Born's rule connects quantum theory to experiment. It is what makes quantum mechanics a scientific theory at
all, able to make predictions that can be tested. ``The Born rule is the crucial link between the abstract mathematical objects of
quantum theory and the world of experience,'' said Lluis Masanes of University College London.

The problem is that Born's rule was not really more than a smart guess --- there was no fundamental reason that led Born to
propose it. ``It was an intuition without a precise justification,'' said Adan Cabello, a quantum theorist at the University of Seville
in Spain. ``But it worked.'' And yet for the past 90 years and more, no one has been able to explain why.

\end{document}
