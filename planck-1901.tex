%\errorcontextlines=10
\documentclass{article}
%\usepackage[dvips]{color}
%\usepackage[dvips]{epsfig}
%\usepackage[english,russian]{babel}
%\RequirePackage{hyperref}
%-------------------------------------------------------------
%
%    editted for ELAN - Digital Anthology of Particle Physics
%    by Kuyanov Yu.V., May 15, 2000
%
%-------------------------------------------------------------
\pdfinfoomitdate=1
\pdftrailerid{}
\begin{document}

M.~Planck, Ann. Phys., {\bf 4}, 553 \hfill {\large \bf 1901}\\

\begin{center}
{\large \bf On the Law of the Energy Distribution\\
in the Normal Spectrum}
\end{center}

\begin{center}
M. Planck\\
(Received January 7, 1901)\\
In other form reported in the German Physical Society 
(Deutsche Physikalische Gesellschaft) in the meetings of
October 19 and December 14, 1900, published in Verh. Dtsch. Phys. Ges.
Berlin, 1900, {\bf 2,} 202 and 237
\end{center}

\centerline{--- ---~~~$\diamond~\diamondsuit~\diamond$~~~--- ---}

\noindent
{\it Translated from German by Kuyanov Yu.V.} 
{[kuyanov@mx.ihep.su]}

\centerline{--- ---~~~$\diamond~\diamondsuit~\diamond$~~~--- ---}

\vspace{0.5cm}

\section*{
{\bf Preface}}

~~~~~The recent spectral measurements of O.~Lummer and 
E.~Pringsheim\footnote
{O.~Lummer, E.~Pringsheim. Verhandl. Deutsch. Phys. Ges., 1900, {\bf 2}, 163.} 
and even more striking those of H.~Rubens and F.~Kurlbaum\footnote{H.~Rubens, 
F.~Kurlbaum. Sitzungsber. Akad. Wiss. Berlin, 1900, 929.}, both confirming 
more recent results obtained by H.~Beckmann\footnote{H. Beckmann. 
Inaug-Dissert. T\"ubingen, 1898, 
see also: H.~Rubens. Wied. Ann., 1899, {\bf 69}, 582.}, 
would discover that the law of the energy distribution in the normal spectrum 
first stated by W.~Wien from the molecular-kinetic consideration and later 
by me from the theory of electromagnetic radiation is not universally correct.

In any case an improvement on the theory is needed 
and I shall further try to carry through 
basing on the theory of electromagnetic radiation developed by me. 
First of all there is necessary for it to find an alterable link 
in the chain of reasons resulting in the Wien's energy distribution law. 
So one handles to remove this link from the chain and 
create a suitable substitute.

The fact that the physical ground of the electromagnetic radiation theory 
including the hypothesis of the ``natural radiation'', 
resists destructive criticism, is shown
in my recent work\footnote{M. Planck. Ann. Phys., 1900, {\bf 1}, 719.};
and since the calculations are known to be error free, 
so the statement remains to be held 
that the energy distribution law of the normal spectrum is totally defined 
if one succeeds in calculation of entropy $S$ of irradiated 
monochromatic vibrating resonator as a function of its vibrational energy. 
So then from the relation $dS/dU = 1/\vartheta$ one keeps the temperature 
$\vartheta$ dependence on energy $U$, 
and since the energy $U$, on the other hand, is simply 
related\footnote{See below equation (8)} 
with a radiation density of appropriate number of vibrations, 
so the temperature dependence on this radiation density is also obtained.
So the normal distribution of energy is one for which the radiation densities 
of any different numbers of vibrations have the same temperature.

Thus the total problem is self reduced to that of 
definition $S$ as a function of $U$, 
and the essential part of the following research is devoted 
to the solution of this problem.
In the first my work on this problem I have entered $S$ directly 
by defining with no further substantiation, as the simple function of $U$, 
and have limited by showing that such form for the entropy 
satisfies to all requirements of the thermodynamics. 
Then I considered that it is alone possible and therefore the Wien's law, 
from it flowing out, necessarily is the universal one. 
In later, more particular 
research\footnote{M.~Planck. Ann. Phys., 1900, {\bf 1}, 730.} 
it seemed to me, however, that it should be expressions, doing the same, 
and that in any case therefore one more condition is needed 
for anyone being able to calculate $S$ uniquely.
It seemed to me that I have found one such condition in the form of statement, 
immediately then considered by me as plausible,
that by the infinitesimal irreversible alteration 
of the near thermal equilibrium being system of $N$ uniform, 
just in stationary radiation field placed resonators, 
the bound up with it alteration of the total entropy $S_N = N S$ 
depends only on their total energy $U_N = NU$ and their alteration 
but not on the energy $U$ of particular resonators.
This statement leads again with necessity 
to the Wien's energy distribution law.
But now however the later is not confirmed by experience, 
so the conclusion is forced 
that this statement in its universality also cannot be right 
and so from the theory is to be removed\footnote{One compares besides 
the criticism, to which this statement is exposed yet: 
W.~Wien. Rapport f\"ur den Pariser Congress, 1900, {\bf 2}, 40; 
O.~Lummer. Loc. cit., p. 92.}.

Therefore yet another condition should be entered 
which enables the calculation of $S$, and for its realization 
the more detailed consideration of the entropy concept is needed. 
The direction of these deliberate thoughts is indicated by 
the consideration of the fragility of early made supposition.
The path is below described, 
in which the new simple expression for entropy 
as well as the new formula for radiation are self found, 
both contradicting no fact established till now.

\section*{
{\bf I. The calculation of entropy of any resonator 
as a function of its energy}}

\vspace{0.5cm}
\subsection*{
{\bf $\S~1$}}
\vspace{0.3cm}

~~~~~An entropy is conditioned by disorder, and this disorder 
in accordance with electro-magnetic theory of radiation 
is based on monochromatic vibrations of any resonator if although 
it remains in a stable stationary field of radiation, on non-regularity 
by which it permanently changes its amplitude and its phase, 
since one clocks time intervals which are long compared with a time 
of vibration, but short compared with a measurement time.
If the amplitude and the phase both are absolutely constant as well as 
vibrations are quite homogeneous, no entropy could exist 
and the vibrational energy should be quite free convertible into the work.
A constant energy $U$ of alone stationary vibrating resonator is therefore 
as an average by time to be perceived or what turns to quite the same result, 
as a simultaneous average of energies of large number $N$ 
of uniform resonators, just into stationary radiation field placed, 
sufficiently removed from one another to have no affect to each other directly.
In this sense in future we will speak about an average energy $U$ 
of a separate resonator. Then a total energy
\begin{equation}
U_N = NU
\end{equation}
of such system of $N$ resonators is corresponded to certain total entropy 
\begin{equation}
S_N = NS
\end{equation}
of the same system where an average entropy of any separate resonator 
is represented by $S$, and this entropy $S_N$ is based on a disorder 
with which the total energy $U_N$ is distributed among particular resonators.

\vspace{0.5cm}
\subsection*{
{\bf $\S~2$}}
\vspace{0.3cm}

~~~~~Now we suppose an entropy $S_N$ 
of a system with an arbitrary remaining additive constant 
to be proportional to logarithm of the probability $W$ 
with which $N$ resonators altogether possess an energy $U_N$; therefore:
\begin{equation}
S_N = k~\mbox{ln}~W + const.
\end{equation}

In my opinion this supposition originates from the base 
of the definition of the probability $W$ mentioned whereas in the premise, 
put on the ground of the electromagnetic theory of radiation, 
we have not any support, enabling to speak about such probability 
in a definite sense.
For the expedience of so aimed supposition its simplicity as well as 
its neighbourhood with that of the kinetic theory of gases are standing for
\footnote{L.~Boltzmann. Sitzungsber d. k. Akad. d. Wissensch. zu Wien (I), 
1877, {\bf 76}, 428.}.

\vspace{0.5cm}
\subsection*{
{\bf $\S~3$}}
\vspace{0.3cm}

~~~~~Now it is worth reminding to find the probability $W$ of $N$ resonators 
alltogether having a vibrational energy $U_N$. 
It is necessary for it to imagine $U_N$ not as a continuous 
unlimited divided value, but as a discrete one, 
composed of integer number of finite equal parts.
If we give a name energy element $\varepsilon$ to such part, 
so one can suppose that
\begin{equation}
U_N = P\cdot\varepsilon,
\end{equation}
where $P$ is an integer, in general, large number, whereas 
the value for $\varepsilon$ is till to be defined.

Now it is clear that the distribution $P$ of energy elements among $N$ 
resonators can happen by some limited quite definite number of manners.
We give a name ``complexion'' to every such manner of distribution 
following L.~Boltzmann who had used this name for an expression 
with a similar idea.
Having numbered resonators by 1, 2, 3, $\ldots, N$, 
one writes them in a row each to another and under each resonator places 
a number of energy elements fallen to it in some arbitrary distribution, 
so for each complexion one obtains a symbol of the following form:

\vspace{0.1cm}

\begin{center}
\begin{tabular}{rrrrrrrrrr}
1~~~&2~~~&3~~~&4~~~&5~~~&6~~~&7~~~&8~~~&9~~~&10\\[0.1cm]
\hline
7~~~&38~~~&11~~~&0~~~&9~~~&2~~~&20~~~&4~~~&4~~~&5\\
\end{tabular}
\end{center}

\vspace{0.2cm}
Here $N = 10$, $P = 100$ are considered. 
The number $\Re$ of all possible complexions is obviously equal to one 
of all possible digital images which can be obtained in this manner 
for the lower row with definite $N$ and $P$.
For intelligibility it should be mentioned that two complexions are considered 
as different if corresponding digital images have the same numbers but in 
a different order. 

Following combinatory, the number of all possible complexions is
$$
\Re = \frac{N\cdot(N + 1)\cdot(N + 2)\ldots(N + P - 1)}
 {1 \cdot 2 \cdot 3 \ldots P} = \frac{(N + P - 1)!}{(N - 1)!~P!}.
$$

Here is in a first approximation according to Stirling offer:
$$
N! = N^N;
$$
therefore in appropriate approximation
$$
\Re = \frac{(N + P)^{N + P}}
{N^N \cdot P^P}.
$$

\vspace{0.5cm}
\subsection*{
{\bf $\S~4$}}
\vspace{0.3cm}

~~~~~The hypothesis, we now wish to put into the base of further 
calculation, is as follows: the probability of that $N$ resonators altogether 
possess vibrational energy $U_N$ is proportional to the number $\Re$ of all 
possible complexions with energy $U_N$ distributed among $N$ resonators, or 
by other words: each certain complexion is as probable as either another one.
It should in last line only by experience be proved
whether this hypothesis virtually hit into nature.
Instead however an opposite one should be possible: 
once an experience should judge in its favor, 
the validity of hypothesis will result in 
the further conclusions on the special nature of resonator's vibrations, 
namely on the character of meanwhile appearing 
``indifferent and in their value compared primary game spaces''
by expression manner of J. v. Kries\footnote{Joh. v. Kries. Die
Principien der Wahrscheinlichkeitsrechnung. Freiburg, 1886, p. 36.}. 
In a modern state of this question a further promotion of this idea 
should certainly appear as premature.

\vspace{0.5cm}
\subsection*{
{\bf $\S~5$}}
\vspace{0.3cm}

~~~~~According to hypothesis introduced in relation with the equation (3), 
the entropy of considered system of resonators with suitable definition 
of additive constant is:
\begin{equation}
S_N = k~ \mbox{ln}~ \Re = k \left\{ \left( N + P \right) \mbox{ln} \left(N + P 
\right) - N~ \mbox{ln}~ N - P ~ \mbox{ln} ~ P \right\},
\end{equation}
and accepting (4) and (1):
$$
S_N = kN \left\{ \left( 1 + \frac{U}{\varepsilon} \right) \mbox{ln} \left( 1 + 
\frac{U}{\varepsilon} \right) - \frac{U}{\varepsilon} ~ \mbox{ln} ~
\frac{U}{\varepsilon} \right\}.
$$

Therefore according to (2), entropy $S$ of a resonator as a function 
of its energy $U$ is: 
\begin{equation}
S = k \left\{ \left( 1 + \frac{U}{\varepsilon} \right) \mbox{ln} 
\left( 1 + \frac{
U}{\varepsilon} \right) - \frac{U}{\varepsilon}~ \mbox{ln} ~\frac{U}
{\varepsilon} \right\}.
\end{equation}

\vspace{0.5cm}
\section*{
{\bf II. The deduction of the Wien's displacement law}}

\vspace{0.5cm}
\subsection*{
{\bf $\S~6$}}
\vspace{0.3cm}

~~~~~Following a Kirchhoff's law of proportionality of both emission- and 
absorbability, discovered by W. Wien
\footnote{W. Wien. Sitzungsber. Acad. Wissensch.
Berlin, 1893, 55.} and called by his name so-called the displacement law,
including, as a particular case, the law of Stefan--Boltzmann 
of full emittance dependence on temperature, 
builds the most valuable constituent 
in the well grounded foundation of the theory of heat radiation.
In a fashion, given by M. Thiesen
\footnote{M. Thiesen. Verhandl. Deutsch. Phys. Ges.,
1900, 2, 66.}, it announces:
$$
E \cdot d \lambda = \vartheta^5 ~ \psi (\lambda \vartheta) \cdot d \lambda,
$$
where $\lambda$ is a wavelength, $Ed \lambda$ is a volume density 
of a spectral slice between $\lambda$ and $\lambda + d \lambda$ 
belonging to ``black'' radiation\footnote{One should perhaps more conveniently 
speak about ``white'' radiation, whose proper generalization 
is now understood as a ``quite white light''.}, 
$\vartheta$ is a temperature and $\psi(x)$ is a known function 
of a single argument $x$. 

\vspace{0.5cm}
\subsection*{
{\bf $\S~7$}}
\vspace{0.3cm}

~~~~~Now we are coming to investigate 
what Wien's displacement law says about our resonator's entropy $S$ 
dependence on its energy and its own period, that is in those general case 
that resonator itself is in an arbitrary diathermal medium.
For this aim first of all let us generalize the Thiesen's form of the law 
on the radiation in an arbitrary diathermal medium with the velocity 
of light propagation $c$.
Since we have to consider not a total radiation but monochromatic one, 
so when comparing different diathermal media, 
the number of vibrations $\nu$ should necessarily be introduced 
instead of wavelength $\lambda$.

Thus the volume density of a spectral slice between $\nu$ and $\nu + d \nu$, 
belonging to energy of radiation, is to be denoted as ${\cal \bf u} d \nu$,  
so one should write: ${\cal \bf u}d \nu$ instead of $E d \nu$, 
$c/\nu$ instead of $\lambda$ and $cd\nu/\nu^2$ instead of $d \lambda$. 
This results in:
$$
{\cal \bf u} = \vartheta^5\cdot\frac{c}{\nu^2}\cdot\psi\left( \frac{c \vartheta}{\nu}
\right).
$$

Now according to known Kirchhoff-Clausius's law, the energy, 
emitted by black surface in a time unit into a diathermal medium, 
for defined temperature $\vartheta$ 
and defined number of vibrations $\nu$ is reverse proportional 
to the square of the velocity of propagation $c^2$; 
thus the volume energy density ${\cal \bf u}$ 
is reverse proportional to $c^3$, and we obtain:
$$
{\cal \bf u} = \frac{\vartheta^5}{\nu^2 c^3} ~f\left( \frac
{\vartheta}{\nu} \right),
$$
where constants of the function $f$ do not depend on $c$.

Instead of it we could also write when $f$ every time, as in following, means 
a new function of a single argument: 
\begin{equation}
{\cal \bf u} = \frac{\nu^3}{c^3} ~ f\left( \frac{\vartheta}{\nu} 
\right)
\end{equation}
and by the way see 
that in a cube of a wavelength size a contained radiation energy 
with a certain temperature as well as a number of vibrations is known to be: 
${\cal \bf u} \lambda^3$, the same for all diathermal media.

\vspace{0.5cm}
\subsection*{
{\bf $\S~8$}}
\vspace{0.3cm}
  
~~~~~In order to pass from the volume density of radiation ${\cal \bf u}$ 
to the energy $U$ of the resonator being in the radiation field and 
stationary vibrating with the same number of vibrations $\nu$, 
we shall use the relation, published in equation (34) of my work 
on non-reversible processes of radiation\footnote{M. Planck. Ann. Phys., 
1900, {\bf 1}, 99.}:
$$
\Re = \frac{\nu^2}{c^2} \cdot U
$$
($\Re$ is the intensity of monochromatic line-polarized beam), 
which together with the known equation
$$
{\cal \bf u} = \frac{8 \pi \Re}{c}
$$
yields the relation: 
\begin{equation}
{\cal \bf u} = \frac{8 \pi \nu^2}{c^3}~U.
\end{equation}

>From here and (7) it follows:
$$
U = \nu f \left( \frac{\vartheta}{\nu} \right),
$$
where now $c$ is not at all present. Instead of it we should also write:
$$
\vartheta = \nu f \left( \frac{U}{\nu} \right).
$$

\vspace{0.5cm}
\subsection*{
{\bf $\S~9$}}
\vspace{0.3cm}

~~~~~Finally introducing yet more the entropy of resonator $S$, 
we assign:
\begin{equation}
\frac{1}{\vartheta} = \frac{dS}{dU}.
\end{equation}
Then it turns out:
$$
\frac{dS}{dU} = \frac{1}{\nu} ~f\left( \frac{U}{\nu} \right)
$$
and integrating, one obtains: 
\begin{equation}
S = f\left( \frac{U}{\nu} \right),
\end{equation}
i.e. the entropy of resonator, vibrating in an arbitrary diathermal medium, 
depends only on the single variable $U/\nu$ and besides keeps only 
the universal constants. 
This, as I know, is the simplest representation of the Wien's displacement law.

\vspace{0.5cm}
\subsection*{
{\bf $\S~10$}}
\vspace{0.3cm}

~~~~~Applying the Wien's displacement law in its latter representation 
to the expression (6) for the entropy $S$, one can realize that 
the energy element $\varepsilon$ should be proportional 
to the number of vibrations $\nu$, so: 
$$
\varepsilon = h\cdot\nu
$$
and therefore: 
$$
S = k \left\{ \left( 1 + \frac{U}{h \nu} \right)~ \mbox{ln}~ \left( 1 + \frac{U}
{h \nu} \right) - \frac{U}{h \nu}~ \mbox{ln} ~\frac{U}{h \nu} \right\}.
$$
Here $h$ and $k$ are the universal constants. 

By substitution into (9) one obtains: 
\begin{equation}
\frac{1}{\vartheta} = \frac{k}{h \nu}~ \mbox{ln} \left( 1 + \frac{h \nu}{U} 
\right),
\end{equation}
$$
U = \frac{h \nu}{e^{\frac{\displaystyle h \nu}{\displaystyle k \vartheta}} - 1}
$$
and the energy distribution law searched then follows from (8):
\begin{equation}
{\cal \bf u} = \frac{8 \pi h \nu^3}{c^3} \cdot \frac{1}{e^{\frac{\displaystyle h
\nu}{\displaystyle k \vartheta}} - 1},
\end{equation}
or also if one with in $\S~7$ shown substitutions 
instead of the number of vibrations $\nu$ 
introduces again the wavelength $\lambda$, that is: 
\begin{equation}
E = \frac{8 \pi ch}{\lambda^5} \cdot \frac{1}{e^{\frac{\displaystyle ch}
{\displaystyle k \lambda \vartheta}} - 1}.
\end{equation}

I suppose to show in the other place the expression for the intensity 
and one for the entropy of the in diathermal medium propagating radiation 
as well as the law of the increase of the total entropy 
in unstationary radiating process.

\section*{
{\bf III. The numeral values}}
\vspace{0.5cm}
\subsection*{
{\bf $\S~11$}}
\vspace{0.3cm}

~~~~~The values of both natural constants $h$ and $k$ may be calculated 
well precisely with a help of measurements available. 
F. Kurlbaum\footnote{F. Kurlbaum. Wied. Ann., 1898, {\bf 65}, 759.} 
has found that if one designates by $S_t$ the total energy, 
radiating into an air in 1 sec from the 1 cm$^2$ surface of the black body 
exposed with $t^{\circ}$, then it is: 
$$
S_{100} - S_0 = 0.0731 ~ \frac{Watt}{cm^2} = 7.31\cdot10^5 ~ 
\frac{erg}{cm^2 \cdot sec}.
$$

>From here the volume density of the total radiation energy in the air 
for the absolute temperature of 1 turns out: 
$$
\frac{4 \cdot 7.31 \cdot 10^5}{3 \cdot 10^{10} \cdot (373^4 - 273^4)} =
7.061 \cdot 10^{-15} ~ \frac{erg}{cm^2 \cdot grad^4}.
$$

>From the other hand, according to (12), the volume density 
of the total radiation energy for $\vartheta = 1$ is as follows: 
$$
u = \int \limits^{\infty}_0~ {\cal \bf u} d \nu = \frac{8 \pi h}{c^3}~ \int
\limits^{\infty}_0~ \frac{\nu^3 d \nu}{e^{\frac{\displaystyle h \nu}
{\displaystyle k}} - 1} 
$$
$$
= \frac{8 \pi h}{c^3}~ \int \limits^{\infty}_0~
\nu^3 \left( e^{-\frac{\displaystyle h \nu}{\displaystyle k}} + 
e^{-\frac{\displaystyle 2 h \nu}{\displaystyle k}} + 
e^{-\frac{\displaystyle 3 h \nu}{\displaystyle k}} + \ldots \right) d\nu
$$
and by all terms integration it yields: 
$$
u = \frac{8 \pi h}{c^3} \cdot 6 \left( \frac{k}{h} \right)^4 \left( 1 + \frac{
1}{2^4} + \frac{1}{3^4} + \frac{1}{4^4} + \ldots \right) =
\frac{48 \pi k^4}{c^3 h^3} \cdot 1.0823.
$$

Assuming it to be equal to $7.061 \cdot 10^{-15}$, one obtains, 
since $c = 3 \cdot 10^{10}$,
\begin{equation}
\frac{k^4}{h^3} = 1.1682 \cdot 10^{15}.
\end{equation}

\vspace{0.5cm}
\subsection*{
{\bf $\S~12$}}
\vspace{0.3cm}

~~~~~O. Lummer and E. Pringsheim\footnote{O. Lummer, E. Pringsheim. 
Verhandl. Deutsch. Phys. Ges., 1900, {\bf 2}, 176.} have determined 
the product $\lambda_m \vartheta$, where $\lambda_m$ is the wavelength 
of the maximum of $E$ in the air for the temperature $\vartheta$, 
having value up to 2940 $\mu \cdot$grad.

So in absolute units that is 
$$
\lambda_m \vartheta = 0.294~ \mbox{cm $\cdot$ grad.}
$$

>From the other hand, if one assumes the partial derivative of $E$ 
in respect to $\lambda$ to be equal to zero, so that $\lambda = \lambda_m$, 
then it follows from (13): 
$$
\left( 1 - \frac{ch}{5k \lambda_m \vartheta} \right) \cdot e^{\frac
{\displaystyle ch}{\displaystyle k \lambda_m \vartheta}} = 1
$$
and from this transcendental equation one obtains: 
$$
\lambda_m \vartheta = \frac{ch}{4.9651 \cdot k}.
$$
Therefore: 
$$
\frac{h}{k} = \frac{4.9651 \cdot 0.294}{3 \cdot 10^{10}} = 4.866 \cdot 10^{-11}.
$$

>From here and from (14) the values for the universal constants turn out: 
\begin{equation}
h = 6.55 \cdot 10^{-27} ~ \mbox{erg $\cdot$ sec,}
\end{equation}

\begin{equation}
k = 1.346 \cdot 10^{-16} ~ \mbox{erg/grad.}
\end{equation}

These are just the same values 
that I have presented in my recent communication.

\end{document}

